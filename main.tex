%%%%%%%%%%%%%%%%%
% This is an sample CV template created using altacv.cls
% (v1.1.3, 30 April 2017) written by LianTze Lim (liantze@gmail.com). Now compiles with pdfLaTeX, XeLaTeX and LuaLaTeX.
% 
%% It may be distributed and/or modified under the
%% conditions of the LaTeX Project Public License, either version 1.3
%% of this license or (at your option) any later version.
%% The latest version of this license is in
%%    http://www.latex-project.org/lppl.txt
%% and version 1.3 or later is part of all distributions of LaTeX
%% version 2003/12/01 or later.
%%%%%%%%%%%%%%%%

%% If you need to pass whatever options to xcolor
\PassOptionsToPackage{dvipsnames}{xcolor}

%% If you are using \orcid or academicons
%% icons, make sure you have the academicons 
%% option here, and compile with XeLaTeX
%% or LuaLaTeX.
% \documentclass[10pt,a4paper,academicons]{altacv}

%% Use the "normalphoto" option if you want a normal photo instead of cropped to a circle
% \documentclass[10pt,a4paper,normalphoto]{altacv}

\documentclass[10pt,a4paper]{altacv}

%% AltaCV uses the fontawesome and academicon fonts
%% and packages. 
%% See texdoc.net/pkg/fontawecome and http://texdoc.net/pkg/academicons for full list of symbols.
%% 
%% Compile with LuaLaTeX for best results. If you
%% want to use XeLaTeX, you may need to install
%% Academicons.ttf in your operating system's font 
%% folder.


% Change the page layout if you need to
\geometry{left=1cm,right=9cm,marginparwidth=6.8cm,marginparsep=1.2cm,top=1.25cm,bottom=1.25cm,footskip=2\baselineskip}

% Change the font if you want to.

% If using pdflatex:
\usepackage[utf8]{inputenc}
\usepackage[T1]{fontenc}
\usepackage[default]{lato}

% If using xelatex or lualatex:
% \setmainfont{Lato}

% Change the colours if you want to
\definecolor{Mulberry}{HTML}{72243D}
\definecolor{SlateGrey}{HTML}{2E2E2E}
\definecolor{LightGrey}{HTML}{666666}
\colorlet{heading}{Sepia}
\colorlet{accent}{Mulberry}
\colorlet{emphasis}{SlateGrey}
\colorlet{body}{LightGrey}

% Change the bullets for itemize and rating marker
% for \cvskill if you want to
\renewcommand{\itemmarker}{{\small\textbullet}}
\renewcommand{\ratingmarker}{\faCircle}

%% sample.bib contains your publications
\addbibresource{mybib.bib}

\begin{document}
\name{Vincent SAMY}
\tagline{Doctorant en robotique}
% \photo{2.8cm}{Globe_High}
\personalinfo{%
  % Not all of these are required!
  % You can add your own with \printinfo{symbol}{detail}
  \email{vsamy@outlook.fr}
  \location{Montpellier, FRANCE}
  \homepage{htpps://vsamy.github.io}
  \linkedin{linkedin.com/in/vsamy}
  \github{github.com/vsamy}
  %% You MUST add the academicons option to \documentclass, then compile with LuaLaTeX or XeLaTeX, if you want to use \orcid or other academicons commands.
%   \orcid{orcid.org/0000-0000-0000-0000}
}

%% Make the header extend all the way to the right, if you want. 
\begin{fullwidth}
\makecvheader
\end{fullwidth}

\cvsection[sidebar]{Publications}

\nocite{*}

\printbibliography[heading=pubtype,title={\printinfo{\faGroup}{Conference Proceedings}},type=inproceedings]

%% Provide the file name containing the sidebar contents as an optional parameter to \cvsection.
%% You can always just use \marginpar{...} if you do
%% not need to align the top of the contents to any
%% \cvsection title in the "main" bar.

\cvsection{Formation}

\cvevent{Doctorat en robotique humanoïde.}{Université de Montpellier}{Oct 2014 -- Oct 2017}{Montpellier, France}
Thèse: \emph{Chute sécurisée de robot humanoïde en environnement encombré} au Lirmm.

\divider

\cvevent{Master en systèmes avancés et robotique}{ENSAM ParisTech}{Sept 2013 -- Juil 2014}{Paris, France}
PFE: \emph{Identification et lois de commande de systèmes partiellement connus} à l’ISIR.
\divider

\cvevent{Diplôme d’ingénieur en génie mécanique}{INSA De Lyon}{Sept 2008 -- Juil 2013}{Villeurbanne, France}
PFE: \emph{Interface à retour de force 2-DDL avec actionneurs hybrides} au LIST - CEA.

\cvsection{Expériences diverses}
\cvevent{Année d'échange}{Université de Kobe}{Sept 2011 -- Juil 2012}{Kobe, Japon}
Projetde recherche réalisé sous la supervision du P\textsuperscript{r} Yokokohji: \emph{Minimisation d’erreurs de mesure des capteurs de vitesse}

\divider

\cvevent{Projet de Conception}{STX France Solutions}{Sept 2012 -- Mars 2013}{Villeurbanne, France}
Intitulé: \emph{Système de mise à l’eau, de récupération et de stockage d’embarcation à partir d’une plateforme Jack-up}

\divider

\cvevent{Auteur, réalisateur, technicien sons et lumières}{Las Semaine Asiatique}{Sept 2009 -- Avr 2010}{Villeurbanne, France}
Réalisation d'un spectacle à vocation culturelle.
\begin{itemize}[leftmargin=7mm]
\item Ecriture de saynètes
\item Gestion d'une team et des artistes du spectacle
\item Filage
\end{itemize}

% \cvsection{Projects}

% \cvevent{Project 1}{Funding agency/institution}{}{}
% \begin{itemize}
% \item Details
% \end{itemize}

% \divider

% \cvevent{Project 2}{Funding agency/institution}{Project duration}{}
% A short abstract would also work.

% \clearpage

%% If the NEXT page doesn't start with a \cvsection but you'd
%% still like to add a sidebar, then use this command on THIS
%% page to add it. The optional argument lets you pull up the 
%% sidebar a bit so that it looks aligned with the top of the
%% main column.
% \addnextpagesidebar[-1ex]{page3sidebar}

\end{document}
